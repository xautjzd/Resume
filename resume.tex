%%%%%%%%%%%%%%%%%%%%%%%%%%%%%%%%%%%%%%%%%%%%%%%%%%%%%%%%%%%%%%%%%%%%%%
% LaTeX Template: Curriculum Vitae
%
% Source: http://www.howtotex.com/
% Feel free to distribute this template, but please keep the
% referal to HowToTeX.com.
% Date: July 2011
% 
%%%%%%%%%%%%%%%%%%%%%%%%%%%%%%%%%%%%%%%%%%%%%%%%%%%%%%%%%%%%%%%%%%%%%%
% How to use writeLaTeX: 
%
% You edit the source code here on the left, and the preview on the
% right shows you the result within a few seconds.
%
% Bookmark this page and share the URL with your co-authors. They can
% edit at the same time!
%
% You can upload figures, bibliographies, custom classes and
% styles using the files menu.
%
% If you're new to LaTeX, the wikibook is a great place to start:
% http://en.wikibooks.org/wiki/LaTeX
%
%%%%%%%%%%%%%%%%%%%%%%%%%%%%%%%%%%%%%%%%%%%%%%%%%%%%%%%%%%%%%%%%%%%%%%

%%%%%%%%%%%%%%%%%%%%%%%%%%%%%%%%%%%%%%%%%%%%%%%%%%%%%%%%%%%%%%%%%%%%%%
% Specify the type of the document
% document type alternative: article,report,book and slides
%%%%%%%%%%%%%%%%%%%%%%%%%%%%%%%%%%%%%%%%%%%%%%%%%%%%%%%%%%%%%%%%%%%%%%
\documentclass[paper=a4,fontsize=11pt]{scrartcl} % KOMA-article class

%%% Call macro sets				
\usepackage[english]{babel}

%%% Character Encode
\usepackage[utf8]{inputenc}                   % 替换你正在使用的编码
\usepackage{CJKutf8}

\usepackage[protrusion=true,expansion=true]{microtype}
\usepackage{amsmath,amsfonts,amsthm}     % Math packages
\usepackage{graphicx}                    % Enable pdflatex
\usepackage[svgnames]{xcolor}            % Colors by their 'svgnames'
\usepackage{geometry}
	\textheight=700px                    % Saving trees ;-)
\usepackage{url}

\frenchspacing              % Better looking spacings after periods
\pagestyle{empty}           % No pagenumbers/headers/footers

%%% Custom sectioning (sectsty package)
%%% ------------------------------------------------------------
\usepackage{sectsty}

\sectionfont{%			            % Change font of \section command
	\usefont{OT1}{phv}{b}{n}%		% bch-b-n: CharterBT-Bold font
	\sectionrule{0pt}{0pt}{-5pt}{3pt}}

%%% Macros
%%% ------------------------------------------------------------
\newlength{\spacebox}
\settowidth{\spacebox}{8888888888}			% Box to align text
\newcommand{\sepspace}{\vspace*{1em}}		% Vertical space macro

\newcommand{\MyName}[1]{ % Name
		\Huge \usefont{OT1}{phv}{b}{n} \hfill #1
		\par \normalsize \normalfont}
		
\newcommand{\MySlogan}[1]{ % Slogan (optional)
		\large \usefont{OT1}{phv}{m}{n}\hfill \textit{#1}
		\par \normalsize \normalfont}

\newcommand{\NewPart}[1]{\section*{\uppercase{#1}}}

\newcommand{\PersonalEntry}[2]{
		\noindent\hangindent=2em\hangafter=0 % Indentation
		\parbox{\spacebox}{        % Box to align text
		\textit{#1}}		       % Entry name (birth, address, etc.)
		\hspace{1.5em} #2 \par}    % Entry value

\newcommand{\SkillsEntry}[2]{      % Same as \PersonalEntry
		\noindent\hangindent=2em\hangafter=0 % Indentation
		\parbox{\spacebox}{        % Box to align text
		\textit{#1}}			   % Entry name (birth, address, etc.)
		\hspace{1.5em} #2 \par}    % Entry value	
		
\newcommand{\EducationEntry}[4]{
		\noindent \textbf{#1} \hfill      % Study
		\colorbox{Black}{%
			\parbox{6em}{%
			\hfill\color{White}#2}} \par  % Duration
		\noindent \textit{#3} \par        % School
		\noindent\hangindent=2em\hangafter=0 \small #4 % Description
		\normalsize \par}

\newcommand{\WorkEntry}[4]{				  % Same as \EducationEntry
		\noindent \textbf{#1} \hfill      % Jobname
		\colorbox{Black}{\color{White}#2} \par  % Duration
		\noindent \textit{#3} \par              % Company
		\noindent\hangindent=2em\hangafter=0 \small #4 % Description
		\normalsize \par}

%%% Begin Document
%%% ------------------------------------------------------------
\begin{document}
\begin{CJK}{UTF8}{gbsn}                       % 详情参阅CJK文件包
% you can upload a photo and include it here...
%\begin{wrapfigure}{l}{0.5\textwidth}
%	\vspace*{-2em}
%		\includegraphics[width=0.15\textwidth]{photo}
%\end{wrapfigure}

\MyName{姜政冬}
%\MySlogan{个人简历}

\sepspace

%%% Personal details
%%% ------------------------------------------------------------
\NewPart{个人信息}{}

\PersonalEntry{手机}{18092482172}
\PersonalEntry{邮箱}{\url{xautjzd@gmail.com}}
\PersonalEntry{地址}{陕西省西安市}
\PersonalEntry{GitHub}{https://github.com/xautjzd}
\PersonalEntry{博客}{http://xautjzd.github.com, http://blog.csdn.net/jiangzhengdong}

%%% Education
%%% ------------------------------------------------------------
\NewPart{教育经历}{}

\EducationEntry{硕士}{2012-至今}{西安理工大学软件工程}{自保研以来,一直不断地给自己充电,参与教研室的项目,并逐渐成为核心开发人员,同时也踊跃参与开源活动。2012年参加OSC西安站活动,并接触Ruby,后自学Ruby,并于2013年7月加入西安Rubists小组,定期参加Rubists小组活动,同时也担任过Rails Girl西安站的教练。}
\sepspace

\EducationEntry{学士}{2008-2012}{西安理工大学网络工程}{大一下自学C++;大二一次偶然接触Linux,自此便深陷其中,只要有空余时间都会钻研,Debian、Red Hat系列都玩过,最终选择Fedora;同时在大二下开始自学Java,直至本科毕业,Java的Struts2/Hibernate/Spring框架都有所研究,本想将Java这条道路走下去,只是由于其他原因最终放弃。}

%%% Skills
%%% ------------------------------------------------------------
\NewPart{个人技能}{}

\SkillsEntry{专业技能}{熟悉Linux系统、Shell编程,熟练使用Vim、Git}
\SkillsEntry{}{熟悉C语言、网络编程,熟悉Java,了解SSH}
\SkillsEntry{}{熟悉ASP.NET MVC熟悉SQL Server,了解MySQL和Mongodb}
\SkillsEntry{}{熟悉HTML/JavaScript/Json/Ajax,熟悉EasyUI/Boostrap}
\SkillsEntry{}{熟悉Ruby及Ruby on Rails,熟悉Markdown}
\SkillsEntry{}{了解函数式编程}
\SkillsEntry{}{具有良好的英文阅读和理解能力}
\SkillsEntry{英语水平}{六级}

%\SkillsEntry{Software}{\textsc{Matlab}, \LaTeX, \textsc{Ansys}, \textsc{Comsol}}

%%% Work experience
%%% ------------------------------------------------------------
\NewPart{项目经历}{}

\EducationEntry{搭建系统框架}{2013.12-至今}{}{将教研室以前着手的几个项目各功能抽取出来,形成一个大体框架,方便今后能够快速地开发,同时将UI从以前的EasyUI转换到Bootstrap,并且在其中集成权限管理、基本增删改查、图表曲线显示、报表生成、地图显示及文件的上传与下载等常用功能。此项工作主要由我进行总体设计,然后按照各功能模块分发给教研室其他人员。}
\sepspace

\EducationEntry{气调库环境监测系统}{2013(5-11)}{}{主要功能为数据转换及数据图表显示。由我独自设计开发,采用ASP.NET MVC3、SQL Server2005、EasyUI和Highcharts进行开发。采集设备将采集到的数据存储在Access数据库中,由于某些原因,需要将Access中的数据导出来,并且以图表形式直观地展示各冷库参数,通过Web进行远程监控,而不用派专职人员蹲守在设备采集点。}
\sepspace

\EducationEntry{生产经营信息管理系统}{2012.8-2013.9}{}{主要功能为信息的录入及报表的生成,采用ASP.NET MVC3、SQL Server2005、EasyUI和 ReportViewer 进行开发。ASP.NET MVC3主要用于项目的构建,同时也会用到Entity Framework,ReportViewer插件则主要用于报表的生成。此系统由我和研三师兄两人负责设计开发,前期由我和师兄共同设计开发,后期由于师兄面临毕业,主要由我一个人负责。}
\sepspace

\EducationEntry{果业技术服务平台}{2011.9-2012.6}{}{主要功能是信息的发布与信息的维护,采用ASP.NET MVC进行开发。从零开始学习ASP.NET MVC,边学习边实践,这段时间进步较大,除熟练掌握ASP.NET MVC外,并逐渐熟悉Ajax/jQuery/HTML等知识。当时ASP.NET MVC刚出不久,中文资料稀缺,所以只能通过英文文档学习,也正是在此环境中锻炼了自己阅读英文文档的能力,同时也逐渐养成学习英文资料的习惯,碰到问题无法解决也是用英文在Google上查找。另外,也感受到了StackOverflow的强大,提升了自己英语提问的能力。}


%%% Awards & Honours
%%% ------------------------------------------------------------
\NewPart{获奖情况}{}

\EducationEntry{被西安秦北学校评为优秀志愿者}{2013}{}

\EducationEntry{被评为2012届优秀毕业生}{2012}{}

\EducationEntry{生产实习中与小组成员一起获得最具团队精神小组称号}{2011}{}

\EducationEntry{获得院图灵杯科技节一等奖}{2010}{}

\EducationEntry{获得国家励志奖学金}{2009-2010}{}

\EducationEntry{获得校三好学生荣誉称号}{2009-2010}{}

\EducationEntry{获得校三等奖学金}{2008-2009}{}

%%% Hobby
%%% ------------------------------------------------------------
\NewPart{兴趣爱好}{}
打羽毛球、乒乓球,看网球,读书,逛博客,关注科技新闻和新技术

%%% Hobby
%%% ------------------------------------------------------------
\NewPart{自我评价}{}
自学能力较强,分析解决问题能力较强。其次,我也一直很注重自己知识的储备,不断地给自己充电,并且规定每月至少阅读一本书籍,每晚至少阅读30分钟。稍微欠缺的可能是沟通能力,属于典型的宅男,不太擅长与陌生人沟通。
\end{CJK}     % 结束中文环境
\end{document}
